\chapter{Forord}
\label{ch:ford}
Denne opgave skrives som afslutning p� �rets p�dagogikum udannelse.

\chapter{Problemformulering}
\label{ch:Prob}

Denne teoretiske p�dagogiske opgave falder under {\bf Tema B: Den faglige progression} 2012, og opgavens problemformulering lyder som f�lger:\vspace{1cm}

{\sl ``Hvad kan kendskab til elevtyper og moderne ledelses samt planl�gningsv�rkt�jer som 4MAT g�re for at sikre den faglige progression.'' }\vspace{1cm}

Opgaven vil behandle tre konkrete emne i forhold til at sikre elevernes faglige progression.
\begin{enumerate}\itemsep=-2pt
	\item Med udgangspunkt i fremherskende teorier af David Kolb \citep[side 175ff, 346f] {Gympd} analyseres l�rings stilstyper og elevtyper med henblik p� at koble dette til personlighedsteori \citep{JTI, MBTI}og hvordan man ved at v�re opm�rksom p� indikatorer kan hj�lpe eleverne fremad.
	\item Opgaven vil ligeledes med udgangspunkt i Steen Becks model og Peter Hobels diskussioner af l�rerens rolle i klasse rummet samt de didaktiske overvejelser sammeholde dette med moderne ledelsesteori, som cooperative learning og situationsbaseret ledelse \citep{Hersey1,Herse2}. Som et middel til at sikre elevernes progression
	\item Sluttelig vil opgaven samle p� p� de to foreg�ende punkter ved at introducere planl�gningsmetoden 4MAT som et redskab der kan styrke den enkelte l�rer i didaktiseringen af enkelt moduler eller forl�b.
\end{enumerate}
Problemformuleringen og de tre cases der her tages op bygger p� en interesse for ledelse og ledelsesteori som v�rkt�j til at hj�lpe elever videre fagligt s�vel som mentalt. Det er mit indtryk at de fleste �ldre l�rer i gymnasieskolen t�nker mindre p� de didaktiske p�dagogiske teorier efterh�nden som de bliver mere og mere tr�nede i at undervise. Dette kan v�re et udtryk for at man er ligeglad med teorien eller at den bliver routine. Jeg er af den holdning at hvis man kan give underviserne nogle v�rkt�jer som virker s� er de ogs� mere villige til at have teorien med i deres daglige arbejde.

%Herunder vil opgaven besk�ftige sig med elevtyper baseret p� David Kolbs teorier (DKT) \citep[side 175ff] {Gympd}. Endvidere vil opgaven inddrage personlighedstype indikatorer baseret p� Jungs teorier (Jungs Type Indicator = JTI) \citep{JTI} samt Myers-Briggs Type indicator (MBTI) \citep{MBTI}. Opgaven vil ogs� inddrage moderne gruppeledelses teorier s� som Cooperative Learning (CL) og situationsbestemt ledelse (SL) \citep{Hersey1,Herse2} sluttelig vil opgaven kigge p� en praktisk oms�ttelse af Kolbs teorier denne kaldes 4MAT modellen.