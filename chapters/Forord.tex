\chapter{Forord}
\label{ch:ford}
Denne opgave skrives som afslutning p� �rets p�dagogikum udannelse.

\chapter{Problemformulering}
\label{ch:Prob}

Opgavens overordnede problemstilling under temaet "Faglig Progression" er:\vspace{1cm}

{\sl Hvad kan kendskab til elevtyper og moderne ledelses samt planl�gningsv�rkt�jer som 4MAT g�re for at sikre den faglige progression.}\vspace{1cm}

Herunder vil opgaven besk�ftige sig med elevtyper baseret p� David Kolbs teorier (DKT) \citep[side 175ff] {Gympd}. Endvidere vil opgaven inddrage personlighedstype indikatorer baseret p� Jungs teorier (Jungs Type Indicator = JTI) \citep{JTI} samt Myers-Briggs Type indicator (MBTI) \citep{MBTI}. Opgaven vil ogs� inddrage moderne gruppeledelses teorier s� som Cooperative Learning (CL) og situationsbestemt ledelse (SL) \citep{Hersey1,Herse2} sluttelig vil opgaven kigge p� en praktisk oms�ttelse af Kolbs teorier denne kaldes 4MAT modellen.