\chapter[Progression og Metode i opgaven]{Progression og Metode}
\label{ch:ProgMet}
Udgangspunktet for denne opgave er et forl�b med titlen ``Verdensbilleder'' (se forl�bsplan \appref{Verden}). Forl�bet er gennemf�rt to gange i forskellige klasser begge 2. �rige B-niveau hold hvor den ene var en 2.g klasse p� tidspunktet for gennemf�rselen mens den anden var en 1.g klasse. Da forl�bene blev gennemf�rt uafh�ngigt af hinanden er der inddraget erfarringer fra det f�rste forl�b i tilrettel�ggelsen af det andet forl�b og modulplanen som er fremlagt i \appref{Verden}. Indledningsvist vil der forekomme en begrebsafklaring samt et afsnit om elevtyper og l�ringsstile for klassen 1.m. P� baggrund af dette vil de didaktiske-teoretiske overvejelser bag forl�bet samt egne overvejelser om elevernes faglige progression blive behandlet.
Opgavens emperi bygger p� observationer og vurderinger som er foretaget i forbindelse med gennemf�rsel af forl�bet.  Indsamlingen og behandlingen af emperi er inspireret af og struktureret i henhold til Bj�rndals Vurderings kube \citep{Bjrndal:2003}. Bj�rndals Vurderingskube best�r af en intentionsdel og en realiseringsdel (se \figref{Kube}). 
\begin{figure}[htb]
	\centering
	\begin{tikzpicture}
		%\draw[very thin, lightgray, step=1mm] (0,0) grid (10,10);
		%\draw[thin, gray, step=5mm] (0,0) grid (10,10);
		%\draw[thick, darkgray, step=10mm] (0,0) grid (10,10);
		
		% DRAW THE CUBE
		\filldraw[fill=blue!10!white, draw=blue!10!black] (1,1) -- (9,1) -- (9,6) -- (1,6) -- cycle;
		\filldraw[fill=blue!10!white, draw=blue!10!black] (9,6) -- (8,8) -- (2,8) -- (1,6) -- cycle; 
		\draw[blue!10!black] (5,1) -- (5,8) (1,3) -- (9,3) (1,4) -- ( 9,4);
		\draw[very thick, black, <->] (1,.75) -- (9,.75);
		\draw[very thick, black, <->] (.75,1) -- (.75,6);
		\node at (5,.5) {Empirisk vurdering};
		\node[rotate=90] at (.5,3.5) {Teoretisk vurdering}; 
		\node[left] at (4,7) {1. Intentioner};
		\node[left] at (7.5,7) {2. Realisering};
		\node[left] at (4.5,5.5) {a) Foruds�tninger for }; 
		\node[left] at (4.7,5) {p�dagogisk praksis};
		\node[left] at (4.7,3.5) {b) P�dagogisk process};
		\node[left] at (4.7,2.5) {c) Resultat af p�dago- }; 
		\node[left] at (3.6,2) {gisk praksis};
		
		\node[left] at (8.5,5.5) {a) Foruds�tninger for }; 
		\node[left] at (8.7,5) {p�dagogisk praksis};
		\node[left] at (8.7,3.5) {b) P�dagogisk process};
		\node[left] at (8.7,2.5) {c) Resultat af p�dago- }; 
		\node[left] at (7.6,2) {gisk praksis};
	\end{tikzpicture}
	\caption[Bj�rndals vurderingskube]{Vuderingskuben af \citet{Bjrndal:2003}}
	\label{fig:Kube}
\end{figure}

 Intentionsdelen forholder sig til de tanker og planer man inddrog i forbindelse med planl�gningen mens realisationen forholder sig til hvad der rentfaktisk skete i  praksis \citep{Bjrndal:2003}. Afslutningsvis vil der v�re en konklusion som s�ger at perspektivere opgaven til den fremtidige uddannelsesm�ssige ramme.
 
\chapter{Begrebsafklaring}
\label{ch:Beg}
\section{L�ring}
\begin{table}
	\centering
	\caption[Kolbs l�ringstilgange]{Kolbs L�ringstilgange \citep[side 347]{Gympd}}
	\begin{tabular}{l l p{7cm}}
		\toprule[2pt]
		Erkendelsesform & L�ringstilgang & Egenskaber hos den l�rende\\
		\midrule
		\multirow{4}{*}{Divergent erkendelse} 	& \multirow{2}{*}{Konkret erfaring} 			& Udviklet forestillingsevne\\ 
										&									& God til at udvikle id\'eer og unders�ge ud fra forskellige perspektiver\\
										& \multirow{2}{*}{Reflekterende observation}	& Interesserer sig for mennesker\\
										&									&  Bredt interessefelt (kulturelt) \\
		\midrule
		\multirow{3}{*}{Assimilativ erkendelse}	& Abstrakt begrebsligg�relse				& Udviklet evne til at danne teoretiske modeller\\
										& \multirow{2}{*}{Reflekterende observation}	& God til induktiiv r�sonnering\\
										&									& Interesse for abstract begreber frem for mennesker\\
		\midrule
		\multirow{4}{*}{Konvergent erkendelse}	& \multirow{2}{*}{Abstrakt begrebsligg�relse}	& St�rk i praktisk anvendelse af id\'eer\\
										&									& God til deduktivt r�sinnement\\
										& \multirow{2}{*}{Aktiv eksperimenteren}		& Ikke f�lelsesbetonet\\
										&									& Sn�vert interessefelt\\
		\bottomrule[2pt]
	\end{tabular}
\end{table}
\label{sec:teach}
\section{Motivation}
\label{sec:mot}

\chapter{L�ringsstile i klassen}
\label{ch:LT}
\section{Elevtyper}
\label{sec:ET}

\chapter[Didaktisk-teoretiske overvejelser om forl�bet]{Didaktisk-teoretiske overvejelser}
\label{ch:DTOF}

\section{Forl�bets teoretiske udgangspunkt}
\label{sec:FTU}

\section{Den faglige progression}
\label{sec:DFP}

\chapter[Analyse og diskussion af den faglige progression]{Analyse og diskussion}
\label{ch:AogD}

\chapter{Evaluering af forl�bet}
\label{ch:Eval}
